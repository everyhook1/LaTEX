\begin{document}

\begin{flushleft}
{\Large \textbf{Fast Protein Loop Sampling and Structure Prediction
Using Distance-Guided Sequential Chain-Growth Monte Carlo Method} }
\\
Ke Tang$^{1}$, Jinfeng Zhang$^{2,\ast}$, Jie Liang$^{1,\ast}$
\\
\bf{1} Department of Bioengineering, University of Illinois at
Chicago, Chicago, Illinois, United States of America
\\
\bf{2} Department of Statistics, Florida State University,
Tallahassee, Florida, United States of America
\\
$\ast$ E-mail: jinfeng@stat.fsu.edu(JZ); jliang@uic.edu (JL)
\end{flushleft}

\section*{Abstract}
Loops in proteins are flexible regions connecting regular secondary
structures. They are often involved in protein functions through
interacting with other molecules. The irregularity and flexibility
of loops make their structures difficult to determine experimentally
and challenging to model computationally. Conformation sampling and
energy evaluation are the two key components in loop modeling.  We
have developed a new method for loop conformation sampling and
prediction based on a chain growth sequential Monte Carlo sampling
strategy, called \underline{Di}stance-guided \underline{S}equential
chain-\underline{Gro}wth Monte Carlo ({\sc DiSGro}). With an energy
function designed specifically for loops, our method can efficiently
generate high quality loop conformations with low energy that are
enriched with near-native loop structures. The average minimum
global backbone RMSD for 1,000 conformations of 12-residue loops is
$1.53$ \r{A}, with a lowest energy RMSD of $2.99$ \r{A}, and an
average ensemble RMSD of $5.23$ \r{A}. A novel geometric criterion
is applied to speed up calculations. The computational cost of
generating these conformations is only about $10$ cpu minutes for
12-residue loops, compared to {\it ca} $180$ cpu minutes using the
FALCm method. Test results on benchmark datasets show that {\sc
DiSGro} performs comparably or better than previous successful
methods, while requiring far less computing time. {\sc DiSGro} is
especially effective in modeling longer loops ($10$--$17$ residues).

\section*{Author Summary}
Loops in proteins are flexible regions connecting regular secondary
structures. They are often involved in protein functions through
interacting with other molecules. The irregularity and flexibility
of loops make their structures difficult to determine experimentally
and challenging to model computationally.  Despite significant
progress made in the past in loop modeling, current methods still
cannot generate near-native loop conformations rapidly.  In this
study, we develop a fast chain-growth method for loop modeling,
called Distance-guided Sequential chain-Growth Monte Carlo ({\sc
DiSGro}), to efficiently generate high quality near-native loop
conformations.  The generated loops can be used directly or as
candidates for further refinement.



\section*{Introduction}
Protein loops connect regular secondary structures and are flexible
regions on protein surface. They often play important functional
roles in recognition and binding of small molecules or other
proteins~\cite{bajorath1996,streaker1999,myllykoski2012}. The
flexibility and irregularity of loops make their structures
difficult to resolve experimentally~\cite{lotan2004}. They are also
challenging to model computationally~\cite{fiser2000,sellers2008}.
Prediction of loop conformations is an important problem and has
received considerable
attention~\cite{van1997,fiser2000,canutescu2003,de2003,depristo2003,michalsky2003,coutsias2004,jacobson2004,zhu2006,zhang2007a,cui2008,sellers2008,spassov2008,liu2009,hildebrand2009,karmali2009,mandell2009,lee2010,zhao2011,arnautova2011,goldfeld2011,subramani2012,fernandez2013}.

Among existing methods for loop prediction, template-free methods
build loop structures {\it de novo}\/ through conformational
search~\cite{bruccoleri1987,van1997,fiser2000,de2003,depristo2003,jacobson2004,zhu2006,sellers2008,spassov2008,liu2009,mandell2009,zhao2011}.
Template-based methods build loops by using loop fragments extracted
from known protein structures in the Protein Data
Bank~\cite{michalsky2003,hildebrand2009,fernandez2013}. Recent
advances in template-free loop modeling have enabled prediction of
structures of long loops with impressive accuracy when crystal
contacts or protein family specific information such as that of GPCR
family is taken into account~\cite{zhu2006,zhao2011,goldfeld2011}.

Loop modeling can be considered as a miniaturized protein folding
problem. However, several factors make it much more challenging than
folding small peptides. First, a loop conformation needs to connect
two fixed ends with desired bond lengths and
angles~\cite{canutescu2003,coutsias2004}. Generating quality loop
conformations satisfying this geometric constraint is nontrivial.
Second, the complex interactions between atoms in a loop and those
in its surrounding make the energy landscape around near-native loop
conformations quite rugged. Water molecules, which are often
implicitly modeled in most loop sampling methods, may contribute
significantly to the energetics of loops. Hydrogen bonding networks
around loops are usually more complex and difficult to model than
those in regular secondary structures. Third, since loops are
located on the surface of proteins, conformational entropy may also
play more prominent roles in the stability of near-native loop
conformations~\cite{zhang2006a,zhang2007b}. Approaches based on
energy optimization, which ignore backbone and/or side chain
conformational entropies, may be biased toward those overly compact
non-native structures. Despite extensive studies in the past and
significant progress made in recent years, both conformational
sampling and energy evaluation remain challenging problems,
especially for long loops ({\it e.g.}, $n \ge 12$).

In this paper, we propose a novel method for loop sampling, called
Distance-guided Sequential chain-Growth Monte Carlo ({\sc DiSGro}).
Based on the principle of chain
growth~\cite{rosenbluth1955,grassberger1997,liu1998,liang2002,zhang2007a},
the strategy of sampling through sequentially growing protein chains
allows efficient exploration of conformational
space~\cite{liu1998,liang2002,zhang2007a,liu2008,zhang2008}. For
example, the Fragment Regrowth by Energy-guided Sequential Sampling
(FRESS) method outperformed previous methods on benchmark HP
sequence folding problem~\cite{zhang2007a,wong2013}. In addition to
HP model~\cite{zhang2007a}, sequential chain-growth sampling has
been used to study protein packing and void
formation~\cite{liang2002}, side chain
entropy~\cite{zhang2004a,zhang2006a}, protein near-native structure
sampling~\cite{zhang2007b}, structure generation from contact
maps~\cite{lin2008}, reconstruction of transition state ensemble of
protein folding~\cite{lin2011}, RNA loop entropy
calculation~\cite{zhang2008}, and structure prediction of
pseudo-knotted RNA molecules~\cite{zhang2009}.

In this study, we first derive empirical distributions of end-to-end
distances of loops of different lengths, as well as empirical
distributions of backbone dihedral angles of different residue types
from a loop database constructed from known protein structures. An
empirical distance guidance function is then employed to bias the
growth of loop fragments towards the $C$-terminal end of the loop.
The backbone dihedral angle distribution is used to sample
energetically favorable dihedral angles, which lead to improved
exploration of low energy loop conformations. Computational cost is
reduced by using a method consisted of exclusion by REsidue-residue
Distance Cutoff and ELLipsoid criterion, called {\sc Redcell}.
Sampled loop conformations, all free of steric clashes, can be
scored and ranked efficiently using an atom-based distance-dependent
empirical potential function specifically designed for loops.

Our paper is organized as follows. We first describe our model and
the {\sc DiSGro} sampling method in detail. We then present results
for structure prediction using five different test data sets. We
show that {\sc DiSGro} has significant advantages in generating
native-like loops. Accurate loops can be constructed by using {\sc
DiSGro} combined with a specifically designed atom-based
distance-dependent empirical potential function. Our method is also
computationally more efficient compared to previous
methods~\cite{de2003,canutescu2003,soto2008,liu2009,lee2010}.

\section*{Results}

\subsection*{Test set}
We use five data sets as our test sets. Test Set 1 contains $10$
loops at lengths four, eight, and twelve, for a total of
$3\times10=30$ loops from $21$ PDB structures, which were described
in Table 2 of Ref.~\cite{canutescu2003}. Test Set 2 consists of $53$
eight, $17$ eleven, and $10$ twelve-residue loops from Table C1 of
Ref.~\cite{soto2008}. Several loop structures were removed as they
were nine-residue loops but mislabeled as eight-residue loops: ({\tt
1awd}, 55--63; {\tt 1byb}, 246--254; and {\tt 1ptf}, 10--18).
Altogether, there are $50$ eight-residue loops. Test Set 3 is a
subset of that of~\cite{fiser2000}, which was used in the RAPPER and
the FALCm studies~\cite{depristo2003,lee2010}. Details of this set
can be found in the ``Fiser Benchmark Set'' section of
Ref.~\cite{depristo2003}. Test Set 4 is taken from Table A1--A6 of
Ref.~\cite{soto2008}. Test Set 5 contains $36$ fourteen, $30$
fifteen, $14$ sixteen and $9$ seventeen-residue loops from Table 3
of Ref.~\cite{zhao2011}. Test Set 1 and 2 are used for testing the
capability of {\sc DiSGro} and other methods in generating
native-like loops. Test Set 3, 4, and 5 are used for assessing the
accuracy of predicted loops based on selection from energy
evaluation using our atom-based distance-dependent empirical
potential function. Our results are reported as global backbone
RMSD, calculated using the N, $\mathrm{C_{\alpha}}$, C and O atoms
in the backbone.

\subsection*{Loop sampling}
To evaluate how good our method is in producing native-like loop
conformations, we use Test Set 1 and 2 to evaluate {\sc DiSGro}.

We generated $5,000$ loops for each of the $10$ loop structures
contained in Test Set 1 at length $4$, $8$, and $12$ residues,
respectively. We compare our results with those obtained by
CCD~\cite{canutescu2003}, CSJD~\cite{coutsias2004},
SOS~\cite{liu2009}, and FALCm~\cite{lee2010}. The minimum RMSD among
$5,000$ sampled loops generated by {\sc DiSGro} are listed in
Table~\ref{tab:dunset}, along with results from the four other
methods.

Accurate loops of longer length are more difficult to generate. For
loops with $12$ residues, {\sc DiSGro} generates more accurate loops
than other methods. Our method has a mean of $1.53$ \r{A}\ for the
minimum RMSD, compared to $1.81$ \r{A}\ for FALCm, the next best
method in the group~\cite{lee2010}. The minimum RMSD of nine of the
ten $12$-residue loops have RMSD $\le2$ \r{A}, while five loops of
the ten targets generated by FALCm have RMSD $>2$ \r{A}. Compared to
the CCD, CSJD, and SOS methods, our loops have significantly smaller
minimum RMSD ($1.53$ \r{A}\ {\it vs}\/ $3.05$, $2.34$, and $2.25$
\r{A}, respectively, Table~\ref{tab:dunset}). The average minimum
global backbone RMSD for $12$-residue loops can be further improved
when we increase the sample size of generated loop conformations.
The minimum global RMSD is improved to $1.45$ \r{A}, $1.26$ \r{A},
and $0.96$ \r{A}\ when the sample size is increased to 20,000,
100,000, and 1,000,000, respectively. Further improvement would
likely require flexible bond lengths and angles.

For loops with $8$ residues, {\sc DiSGro} has an average minimum
RMSD value smaller than the CCD, CSJD, and SOS methods ($0.81$
\r{A}\ {\it vs}\/ $1.59$ \r{A}, $1.01$ \r{A}, and $1.19$ \r{A},
respectively, Table~\ref{tab:dunset}). In eight of the ten
$8$-residue loops, {\sc DiSGro} achieves sub-angstrom accuracy (RMSD
$<1$ \r{A}), although the mean of minimum RMSD of $8$-residue loops
is slightly larger than that from FALCm ($0.80$ \r{A}\ {\it vs}\/
$0.72$ \r{A}).

For loops with $4$-residue, the mean of the minimum RMSD ($0.21$
\r{A})
 by {\sc DiSGro} is significantly smaller
than those by the CSJD and the CCD methods ($0.40$ \r{A}\ and $0.56$
\r{A}, respectively), and is similar to those by the SOS and FALCm
methods($0.20$ \r{A}\ and $0.22$ \r{A}, respectively). Noticeably,
three of the ten loops have RMSD $<0.1$ \r{A}, indicating our
sampling method has good accuracy for short loop modeling.

These loops can be generated rapidly. The computing time per
conformation averaged over 5,000 conformations for $4$, $8$, and
$12$-residues is $4.4$, $13$, and $20\:ms$ using a single AMD
Opteron processor of $2\:GHz$. In addition to improved average
minimum RMSD, {\sc DiSGro} seems to take less time than CCD ($31$,
$37$, and $23\:ms$ on an AMD 1800+ MP processor for the $4$, $8$,
and $12$-residue loops), and is as efficient as SOS ($5.0$, $13$,
and $19\:ms$ for the $4$, $8$, and $12$-residue loops on an AMD
1800+ MP processor).

Reducing the number of trial states in {\sc DiSGro} can further
reduce the computing time, with some trade-off in sampling accuracy.
For example, when we take $(m, \, n)  = (10, \,2)$, the computing
time per conformation averaged over 5,000 conformations for $4$,
$8$, and $12$-residues is only $3.5$, $5.0$, and $5.8\:ms$,
respectively, with the average minimum RMSDs comparable to those
from SOS's ($0.29$ \r{A}\ {\it vs}\/ $0.20$ \r{A}, $1.15$ \r{A}\
{\it vs}\/ $1.19$ \r{A}, and $2.24$ \r{A}\ {\it vs}\/ $2.25$ \r{A}\
for the $4$, $8$, and $12$-residue loops, respectively). Although
the CSJD loop closure method has faster computing time ($0.56$,
$0.68$, and $0.72\:ms$ on AMD 1800+ MP processor), the speed of {\sc
DiSGro} is adequate in practical applications.

We compare {\sc DiSGro} in generating near-native loops with
Wriggling~\cite{cahill2003}, Random Tweak~\cite{shenkin1987}, Direct
Tweak~\cite{xiang2002,soto2008},
$\mathrm{LOOPY_{bb}}$~\cite{xiang2002}, and
PLOP-build~\cite{jacobson2004} using Test Set 2. The minimum RMSD
among $5,000$ loops generated by {\sc DiSGro} are listed in
Table~\ref{tab:sotobuilder}, along with results from the other
methods obtained from Table 2 in Ref.~\cite{soto2008}. Direct Tweak
and $\mathrm{LOOPY_{bb}}$ from the LoopBuilder method and our {\sc
DiSGro} have better accuracy in sampling than Wriggling, Random
Tweak, and PLOP-build methods.  For loops with 11 and 12-residues,
these three methods are the only ones that can generate near-native
loop structures with minimal RMSD values below $2$ \r{A}. Among
these, {\sc DiSGro} outperforms $\mathrm{LOOPY_{bb}}$ in generating
loops at all three lengths: the average minimal RMSD ($R_{\rm
  min}$) is
  $1.28$ \r{A}\ {\it vs.}\ $1.80$ \r{A}\ for length $12$,
  $1.19$ \r{A}\ {\it vs.}\ $1.51$ \r{A}\ for length $11$, and
  $0.80$ \r{A}\ {\it vs.}\ $0.89$ \r{A}\ for length $8$,
  respectively. Compared to the Direct
  Tweak sampling method, {\sc DiSGro} has improved $R_{\rm min}$ for
$12$-residue loops ($1.28$ \r{A}\ {\it vs}\/ $1.48$ \r{A}), slightly
improved $R_{\rm min}$ for $11$-residue loops ($1.19$ \r{A}\ {\it
vs}\/ $1.20$ \r{A}) and inferior $R_{\rm min}$ for $8$-residue loops
($0.80$ \r{A}\ {\it vs}\/ $0.69$ \r{A}).  Overall, these results
show that {\sc DiSGro} are very effective in sampling near-native
loop conformations, especially when modeling longer loops of length
11 and 12.

Our {\sc DiSGro} method can generate accurate loops and has
significant advantages for longer loops compared to previous
methods. Using RMSD values calculated from three backbone atoms N,
$\mathrm{C_{\alpha}}$, and C for all loop lengths lead to the same
conclusion.


\subsection*{Loop structure prediction and energy evaluation}
To assess the accuracy of predicted loops based on selection from
energy evaluation using our specifically designed atom-based
distance-dependent empirical potential function, we test {\sc
DiSGro} using Test Set 3 and follow the approach of
reference~\cite{lee2010} for ease of comparison. Because of the high
content of secondary structure, these loops are very challenging to
model. In the study of~\cite{lee2010}, $1,000$ backbone
conformations with the best scores of the DFIRE potential
function~\cite{zhou2002} were retained after screening $4,000$
generated backbone conformations for each loop. Loop closure and
steric clash removal were not enforced to the $4,000$ conformations.
We follow the same procedure, except the DFIRE potential function is
replaced by our atom-based distance-dependent empirical potential
function. The ensemble of the selected $1,000$ backbone
conformations are then subjected to the procedure of side-chain
construction as described in the Section ``Side-chain modeling and
steric clash removal''. The loop conformations with full side-chains
are then scored and ranked by the atom-based distance-dependent
empirical potential function. Our results are summarized in
Table~\ref{tab:falcm}.

We measure the average minimum backbone RMSD $R_{\rm min}$, the
average ensemble RMSD $R_{\rm ave}$, and the average RMSD of the
lowest energy conformations $R_{Emin}$ of the 1,000 loop ensemble
with the same length. Overall, {\sc DiSGro} performs significantly
better than FALCm and RAPPER in $R_{\rm min}$, $R_{\rm ave}$ and
$R_{Emin}$ for all loop lengths. Compared to FALCm, {\sc DiSGro}
shows significant advantages in $R_{\rm min}$ on sampling long loops
of $10$--$12$ residues. Our method has $R_{\rm min}$ of $1.15$
\r{A}\ compared to $1.45$ \r{A}\ for $10$-residue loops, $1.39$
\r{A}\ compared to $1.47$ \r{A}\ for $11$-residue loops, and $1.53$
\r{A}\ compared to $1.74$ \r{A}\ for $12$-residue loops,
respectively. For example, as can be seen in Figure~\ref{fig:pdb},
the lowest energy loop (red) of a 12-residue loop in the protein
{\tt 1scs} (residues 199-210) has a $0.9$ \r{A}\ RMSD to the native
structure (white). The generated top five lowest energy loops are
all very close to the native loop, yet are diverse among themselves.

{\sc DiSGro} also generates loops with smaller $R_{\rm ave}$
compared to FALCm in loops with length ranging from $4$ to $12$,
indicating {\sc DiSGro} can generate ensemble of loop conformations
with enriched near native conformations. Furthermore {\sc DiSGro}
achieves better modeling accuracy using the atom-based
distance-dependent empirical potential function. Compared to FALCm,
{\sc DiSGro} has a $R_{Emin}$ of $1.72$ \r{A}\ {\it vs}\/ $1.87$
\r{A}\ for $8$-residue loops, $1.82$ \r{A}\ {\it vs}\/ $2.08$ \r{A}\
for $9$-residue loops, $2.33$ \r{A}\ {\it vs}\/ $3.09$ \r{A}\ for
$10$-residue loops, $2.98$ \r{A}\ {\it vs}\/ $3.43$ \r{A}\ for
$11$-residue loops, and $2.99$ \r{A}\ {\it vs}\/ $3.84$ \r{A}\ for
$12$-residue loops, respectively.

{\sc DiSGro} is also much faster than other methods. The reported
typical computational cost of FALCm is $180$ cpu minutes for
$8$--$12$ residue loops on a Linux server of a $2.8\:GHz$ 2-core
Intel Xeon processor~\cite{ko2011}. The computation cost for {\sc
DiSGro} method is only $6$ and 10 cpu minutes for $10$ and
12--residue loops on a single $2\:GHz$ AMD Opteron processor,
respectively. In addition, FALCm has a size restriction, and it only
works with proteins with $<500$ residues. In contrast, the overall
protein size has no effect on the computational efficiency of {\sc
DiSGro} once all of the rest of the protein residues have been
tested using the ellipsoid criterion.

The LOOPER method is an accurate and efficient loop modeling method
using a minimal conformational sampling method combined with energy
minimization~\cite{spassov2008}. The test set used in the LOOPER
study are the original Fiser data set without removal of any loops.
Therefore, it is different from Test Set 3 used in the RAPPER and
FALCm studies~\cite{depristo2003,lee2010}. For ease of comparison,
we compare our {\sc DiSGro} method to the LOOPER method using the
test set with $10$--$12$-residue loops from~\cite{spassov2008}. Our
results are summarized in Table~\ref{tab:looper}.

We denote $R_{\rm Bkb,ave}$ and $R_{\rm Bkb,med}$ as the mean and
median of backbone RMSD of the lowest energy conformations with the
same loop length. Similarly, we use $R_{\rm Atm,ave}$, and $R_{\rm
Atm,med}$ to denote the mean and median RMSD values of all-heavy
atoms RMSD. {\sc DiSGro} shows improved prediction accuracy compared
to LOOPER in both backbones and all-heavy atoms RMSD. For the $40$
loops of length 12, $R_{\rm Bkb,ave}$ is $3.20$ \r{A}\ compared to
$4.08$ \r{A}, while the median $R_{\rm Bkb,med}$ is $2.39$ \r{A}\
compared to $3.80$ \r{A}. It also has better all-heavy atom RMSD of
$3.39$ \r{A}/$3.18$ \r{A}\ (mean/median), compared to $3.58$
\r{A}/$3.35$ \r{A}\ for $10$-residue loops, $3.58$ \r{A}/$3.30$
\r{A}\ compared to $4.30$ \r{A}/$3.60$ \r{A}\ for $11$-residue
loops, and $4.18$ \r{A}/$3.60$ \r{A}\ compared to $5.22$
\r{A}/$4.96$ \r{A}\ for $12$-residue loops.

It is worth noting that {\sc DiSGro} outperforms LOOPER in speed as
well. For a loop with $10$ residues, the time cost of {\sc DiSGro}
is $6$ minutes using a $2\:GHz$ CPU versus $40$ cpu minutes using a
$3\:GHz$ processor according to Figure~$7$ in the LOOPER
paper~\cite{spassov2008}.

Prior publications also allowed us to compare results in loop
structure predictions based on energy discrimination using Test Set
4 with results obtained using  the LoopBuilder
method~\cite{soto2008}. Following~\cite{soto2008},
 we generated $1,000$ closed loop conformations for
eight-residue loops, $2,000$ for nine-residue loops, $5,000$ for
ten, eleven, and twelve-residue loops, and $8,000$ for
thirteen-residue loops. Energy calculations are carried out using
our atom-based distance-dependent empirical potential function. The
average RMSD of the lowest energy conformations, $R_{Emin}$, are
then compared between these two methods. The results are summarized
in Table~\ref{tab:sotobuilderE}.

Compared to LoopBuilder, {\sc DiSGro} has better $R_{Emin}$: $1.83$
\r{A}\ {\it vs}\/ $1.88$ \r{A}\ for $9$-residue loops, $1.83$ \r{A}\
{\it vs}\/ $1.93$ \r{A}\ for $10$-residue loops, $2.38$ \r{A}\ {\it
vs}\/ $2.50$ \r{A}\ for $11$-residue loops, $2.62$ \r{A}\ {\it vs}\/
$2.65$ \r{A}\ for $12$-residue loops, and $3.26$ \r{A}\ {\it vs}\/
$3.74$ \r{A}\ for $13$-residue loops, respectively. {\sc DiSGro} has
inferior performance in selecting $R_{Emin}$ for $8$-residue loops
($1.59$ \r{A}\ {\it vs}\/ $1.31$ \r{A}). The average time using
LoopBuilder for twelve-residue loops was around 4.5 hours or 270
minutes, while the computational time using {\sc DiSGro} is around
10 minutes. Overall, {\sc DiSGro} has equal or slightly better
performance than LoopBuilder in average prediction accuracy of loop
structures with far less computing time.

To test the feasibility of {\sc DiSGro} in modeling longer loops
with length $>12$, we use the Fiser $13$-residue loops data set to
generate and select low energy loop conformations. $1,000$
conformations with low energy are obtained. The mean of minimum
backbone RMSD $R_{\rm min}$ of $40$ loops with $13$-residue is
$1.76$ \r{A}, and the median is $1.61$ \r{A}. The mean/median of the
backbones RMSD $R_{Bkb,Emin}$, and all heavy atoms RMSD $R_{\rm
Atm,Emin}$ of the lowest energy conformations are $2.91$
\r{A}/$2.53$ \r{A}\ and $3.84$ \r{A}/$3.29$ \r{A}, respectively
(Table~\ref{tab:fiser13}).

With extensive conformational sampling using molecular mechanics
force field, the Protein Local Optimization Program (PLOP) can
predict highly accurate loops~\cite{jacobson2004,zhu2006,zhao2011}.
We tested {\sc DiSGro} using Test Set 5 consisting of $89$ loops
with length $14$--$17$ and compared results with those using PLOP.
Here the sampling and scoring processes were similar to those used
in Test Set 3, except 100,000 backbone conformations were generated.
We measured the average minimum backbone RMSD $R_{\rm min}$ and the
average RMSD of the lowest energy conformations $R_{Emin}$. Our
results are summarized in Table~\ref{tab:zhaoplop}.

Loops predicted by the PLOP method have smaller $R_{E_{\min}}$
compared to {\sc DiSGro}~\cite{zhao2011}, although {\sc DiSGro}
samples well and gives small $R_{\min}$ of $1.58$ \r{A}\ for
$14$-residue loops, $1.80$ \r{A}\ for $15$-residue loops, $1.88$
\r{A}\ for $16$-residue loops, and $2.18$ \r{A}\ for $17$-residue
loops. For loops of length $17$, the $R_{\rm min}$ of $2.18$ \r{A}\
is less than the reported $R_{Emin}=2.30$ \r{A}\ using PLOP,
although it is unclear whether the $R_{\rm min}$ of loops generated
by PLOP is less than $2.18$ \r{A}. Overall, {\sc DiSGro} is capable
of successfully generating high quality near-native long loops, up
to length 17. The accuracy of $R_{Emin}$ of loops generated by {\sc
DiSGro} may be further improved by using a more effective scoring
function.

We also compared the computational costs of the two methods. The
average computing time for {\sc DiSGro} is $0.73$, $0.72$, $0.81$,
and $0.95$ hours for loops of lengths $14$, $15$, $16$, and $17$
using a single core AMD Opteron processor $2350$, respectively,
which is more than two orders of magnitude less than the time
required for the PLOP method ($216.0$, $309.6$, $278.4$, and $408.0$
hours for loops of length $14$, $15$, $16$, and $17$ residues,
respectively).


\subsection*{Improvement in computing efficiency}

We used a REsidue-residue Distance Cutoff and ELLipsoid criterion
({\sc Redcell}) to improve the computational efficiency. To assess
the effectiveness of this approach, we carry out a test using a set
of 140 proteins (see discussion of the tuning set in Materials and
Methods). We compared the time cost of energy calculation of
generating a single loop, with and without this procedure. When the
procedure is applied, we only calculate the pairwise atom-atom
distance energy between atoms in loop residues and other atoms
within the ellipsoid. When the procedure is not applied, we
calculate energy function between atoms in loop residues and all
other atoms in the rest of the protein. The computational cost of
energy calculations for sampling single loops with $12$ and
$6$-residues are shown in Figure~\ref{fig:ellipresult}A and
Figure~\ref{fig:ellipresult}B, respectively.

From Figure~\ref{fig:ellipresult}, we can see that significant
improvement in computational cost is achieved. The average time cost
using our procedure is reduced from $82.3\:ms$ to $6.0\:ms$ for
sampling $12$-residue loops, and $39.4\:ms$ to $2.0\:ms$ for
$6$-residue loops. In addition, this approach makes the time cost of
energy calculations independent of the protein
size~(Figure~\ref{fig:ellipresult}A and
Figure~\ref{fig:ellipresult}B), whereas the computing time without
applying this procedure increases linearly with the protein size.
The improvement is especially significant for large proteins. For
example, to generate a $6$-residue loop in a protein with $1,114$
residues, the computing time is improved from $93.7\:ms$ to
$1.8\:ms$, which is more than $50$-fold speed-up. Detailed
examination indicates that both distance cutoff and the ellipsoid
criterion contribute to the computing efficiency. Furthermore, the
full {\sc Redcell} procedure has improved efficiency over using
either ``Ellipsoid Criterion Only'' or ``Cutoff Criterion Only''.
The computing time for generating a $6$-residue loops is $2.0\:ms$
when the full {\sc Redcell} procedure is applied, compared to
$5.3\:ms$, and $3.9\:ms$, when only the ellipsoid criterion and only
the distance-threshold are used, respectively
(Figure~\ref{fig:ellipresult}C). Furthermore, there is no loss of
accuracy in energy evaluation. Overall, {\sc Redcell} improves the
computational cost by excluding many atoms from collision detections
and energy calculations, with significant reduction in computation
time, especially for large size proteins.



\section*{Discussion}
In this study, we presented a novel method Distance-guided
Sequential chain-Growth Monte Carlo ({\sc DiSGro}) for generating
protein loop conformations and predicting loop structures. High
quality ensembles of loops with enriched near-native conformations
can be efficiently generated using the {\sc DiSGro} method. {\sc
DiSGro} has better average minimum backbone RMSD $R_{\min}$ compared
to other loop sampling methods. For example, $R_{\min}$ is $1.53$
\r{A}\ for 12-residue loops when using {\sc DiSGro}, but takes the
value of $3.05$ \r{A}, $2.34$ \r{A}, $2.25$ \r{A}, and $1.81$ \r{A}\
when using the CCD, CSJD, SOS, and the FALCm method.

{\sc DiSGro} also performs well in identifying native-like
conformations using atom-based distance-dependent empirical
potential function. In comparison with other similar loop modeling
methods, {\sc DiSGro} demonstrated improved modeling accuracy, with
an average RMSD of the lowest energy conformations $R_{Emin}$ for
the more challenging task of sampling longer loops of $10$-$13$
residues. For example, {\sc DiSGro} outperforms FALCm~\cite{lee2010}
($2.33$ \r{A}\ {\it vs}\/ $3.09$ \r{A}) and
LOOPER~\cite{spassov2008} ($2.30$ \r{A} {\it vs}\/ $2.66$ \r{A}) in
predicting $10$-residue loops, while taking less computing time ($6$
minutes {\it vs\/} $180$ minutes for FALCm and $40$ minutes for
LOOPER. Compared to LoopBuilder~\cite{soto2008}, {\sc DiSGro} also
has better $R_{Emin}$: For $13$-residue loops, the $R_{Emin}$ is
$3.26$ \r{A}\ using {\sc DiSGro}, but is $3.74$ \r{A}\ when using
the Loop Builder. The average computing time is also faster when
using {\sc DisGRo}: it takes about $6$ minutes to predict structures
of $10$-residue loops and $10$ minutes for $12$-residue loops. {\sc
DiSGro} also works well for short loops, although this may be
largely a reflection of the underlying analytical closure
method~\cite{coutsias2004}.

There are a number of directions for further improvement. {\sc
DiSGro} can be further improved by adding fragments of peptides when
growing loops instead of adding individual residues. Fragment-based
approach has been widely used in protein structure
prediction~\cite{simons1997,rohl2004,sheffler2010,leaver2011} and
specifically in loop structure prediction~\cite{mandell2009}. It is
straightforward to apply the strategy described in this study for
fragment-based growth, and it will likely lead to improved sampling
efficiency further and enable longer loops to be modeled.
Furthermore, the energy function employed here can be further
improved by optimization such as those obtained by training with
challenging decoy loops using nonlinear kernel~\cite{hu2004}, and/or
using rapid iterations through a physical convergence
function~\cite{thomas1996,huang2011}. In addition, {\sc DiSGro} is
compatible with different loop closure
methods~\cite{canutescu2003,coutsias2004,lee2010}, and experimenting
with other closure strategy may also lead to further improvement.

An efficient loop sampling method such as {\sc DiSGro} can help to
improve overall modeling of loop structures. Currently, the
hierarchical approach of the Protein Local Optimization Program
(PLOP)~\cite{jacobson2004,zhu2006,zhao2011} gives excellent accuracy
in protein loop modeling, but requires significant computational
time. The average time cost of modeling a $13$-residue loop is about
4-5 days~\cite{zhao2011}. Kinematic closure (KIC) method can also
make very accurate predictions of $12$-residue
loops~\cite{mandell2009}. However, KIC also requires substantial
computation, with about $320$ CPU hours on a single $2.2\:GHz$
Opteron processor for predicting 12-residue
loops~\cite{mandell2009}. As suggested earlier by Spassov et
al~\cite{spassov2008}, an efficient loop modeling method combined
with energy minimization may overcome the obstacle of high
computational cost. By generating high quality initial structures
using {\sc DiSGro}, near native conformations of loops can be used
as candidates for further refinement.

\section*{Materials and Methods}

\subsection*{Protein structures representation}
All heavy atoms in the backbone and side chain of a protein loop are
explicitly modeled. The bond lengths $b$ and angles $\theta$ are
taken from standard values specific to residue and atom
type~\cite{Engh1991}. The backbone dihedral angles
$(\phi,\psi,\omega)$ and side chain dihedral angles $\chi$
constitute all the degrees of freedom (DOFs) in our model.

\subsection*{Distance-guided Sequential chain-Growth Monte Carlo ({\sc DiSGro})}
In order to efficiently generate adequate number of native-like loop
conformations, we have developed a Distance-guided Sequential
chain-Growth Monte Carlo ({\sc DiSGro}) method.

Let the loop to be modeled begins at residue $t$ and ends at residue
$l$. The sequence of the positions of backbone heavy atoms from $C$
atom of residue $t$ to $C_\alpha$ ($CA$) atom of residue $l$ are
unknown and need to be generated. We assume that the backbone atoms
before and after this fragment are known. Coordinates of side chain
atoms are also unknown and need to be generated if the coordinates
of the $CA$ atoms they are attached to are unknown.

At each step of the chain growth process, we generate three
consecutive backbone atoms continuing from the backbone atom sampled
at the previous step. At the $(i-t)$-th growth step ($t \leq i <
l$), the three backbone atoms are $C$ atom of residue $i$, $N$ atom
of residue $i+1$, and $C_\alpha$ atom of residue
$i+1$~(Figure~\ref{fig:phi-psi}). The coordinates of the three
atoms, $C_{i}$, $N_{i+1}$ and $CA_{i+1}$, are denoted as
$\mathbf{x}_{C,i}$, $\mathbf{x}_{N, i+1}$, and $\mathbf{x}_{CA,
i+1}$, respectively. The $\omega$ dihedral angles that determine the
coordinate of $C_\alpha$ atoms are sampled from a normal
distribution with mean $180^{\circ}$ and standard deviation
$4^{\circ}$. In the next section, we describe in detail in sampling
of the dihedral angles $(\phi, \psi)$, which determine the
coordinates of the $C$ and the $N$ atoms.

\subsubsection*{Sampling backbone $(\phi, \psi)$ angles}
Without loss of generality, we describe the sampling procedure for
$C_i$ and $N_{i+1}$ atoms at the $(i-t)$-th growth step. $C_i$ is
generated first, followed by $N_{i+1}$. Denote the distance between
$\mathbf{x}_{CA,i}$ and $\mathbf{x}_{C,l}$ as $d_{CA_i, C_l}
 = |\mathbf{x}_{C,l} - \mathbf{x}_{CA,i}|$, and the
distance between $\mathbf{x}_{C,i}$ and $\mathbf{x}_{C,l}$ as
$d_{C_i, C_l} = | \mathbf{x}_{C,i} - \mathbf{x}_{C,l}|$. Since the
bond angle $\theta_{C,i}$ formed by the $N_i-CA_i$ and $CA_i - C_i$
bonds is fixed, and the bond length $b_{CA_i,C_i}$ is also fixed,
$C_i$ will be located on a circle $\mathbf{C_C}$~(
Figure~\ref{fig:phi-psi}):
\begin{equation}
\label{eqn:circle-C} \mathbf{C_C} = \{\mathbf{x} \in \mathbb{R}^3 |
\mbox{ such that }||\mathbf{x} - \mathbf{x}_{CA,i}|| = b_{CA_i,C_i}
\mbox{ and } (\mathbf{x} - \mathbf{x}_{CA,i}) \cdot
(\mathbf{x}_{CA,i} - \mathbf{x}_{N,i}) = \cos \theta_{C,i} \}.
\end{equation}
Given a fixed $d_{C_i, C_l}$, $C_i$ can be placed on two positions
$\mathbf{x}_{C,i}$ and $\mathbf{x}_{C',i}$ on circle
$\mathbf{C_C}$~(Figure~\ref{fig:phi-psi}, $\mathbf{x}_{C,i}$ and
$\mathbf{x}_{C',i}$ are labeled as $C_i$ and $C'_i$, respectively.)
As the probability for placing $C_i$ on either position is about
equal based on our analysis, we randomly select one position to
place atom $C_i$.

In principle, sampling from the empirical distributions of $d_{C_i,
C_l}$ and mapping back to $C_i$ should encourage the growth of loops
to connect to the terminal $C_l$ atom. Further analysis of the
empirical distribution of $d_{C_i,C_l}$ given $d_{CA_i,C_l}$ shows
that $d_{CA_i,C_l}$ can be very informative for sampling
$d_{C_i,C_l}$ in some cases. This lead us to design the sampling of
$\mathbf{x}_{C_i}$ based on the conditional distribution of
$\pi(d_{C_i,C_l} | d_{CA_i,C_l})$. See below for details.

Generating atom $N_{i+1}$ is similar to generating $C_i$, only
$N_{i+1}$ instead of $C_i$ is placed on a circle $\mathbf{C_N}$:
\begin{equation}
\label{eqn:circle-N} \mathbf{C_N} = \{\mathbf{x} \in \mathbb{R}^3 |
\mbox{ such that }||\mathbf{x} - \mathbf{x}_{C, i}|| =
b_{C_i,N_{i+1}} \mbox{ and } (\mathbf{x} - \mathbf{x}_{C, i}) \cdot
(\mathbf{x}_{C, i} - \mathbf{x}_{CA,i}) = \cos \theta_{N,i+1} \},
\end{equation}
where $b_{C_i,N_{i+1}}$ is the bond length between atom $C_i$ and
atom $N_{i+1}$, and the distance between $\mathbf{x}_{N,i+1}$ and
$\mathbf{x}_{C,l}$ is $d_{N_{i+1}, C_l} = | \mathbf{x}_{N,i+1} -
\mathbf{x}_{C,l}|$. Similarly, atom $N_{i+1}$ is placed by sampling
$d_{N_{i+1},\, C_l}$ condition on $ d_{C_i,C_l}$ from the empirical
conditional density $\pi(d_{N_{i+1},C_l} | d_{C_i, C_l})$. We repeat
this process $m$ times to generate $m$ trial positions of $C_i$,
$N_{i+1}$, and $CA_{i+1}$.


\subsubsection*{Sampling $d_{C_i,C_l}$ and $d_{N_{i+1},C_l}$ from conditional distributions}

We sample $d_{C_i,C_l}$ from the conditional distribution
$\pi(d_{C_i,C_l} | d_{CA_i,C_l})$ to obtain the location of $C_i$
atom. We first construct the empirical joint distribution
$\pi(d_{CA_i,C_l}, d_{C_i,C_l})$ by collecting $(d_{CA_i,C_l},
d_{C_i,C_l})$ pairs over all loops in a loop database derived from
the CulledPDB database (version 11118, at 30\% identity, 2.0\r{A}\
resolution, and with $R=0.25$)~\cite{wang2003}. From the 6,521
protein structures in the CulledPDB, we remove $7$ PDB structures
which appear in our test data set. For the rest of $6,514$ protein
structures, loop regions were identified using the secondary
structure information either directly from the PDB records or from
classification provided by the DSSP software~\cite{kabsch1983}. All
random coil regions, including $\alpha$-helices and $\beta$-strands
with length $<4$ amino acids, are included in our database. In
total, we have $49,336$ loop structures.

For each set of loops with the same residue separation $(l-i)$,
$(d_{CA_i,C_l}, d_{C_i,C_l})$ are Winsorised at $99.9\%$
level~\cite{lewis2008}. Specifically, the extreme values above
$99.9\%$ are replaced by the values at the $99.9$ percentile. We
then use a nonparametric two-dimensional Gaussian kernel density
estimator to construct a smooth bivariate distribution
$\pi(d_{CA_i,C_l}, d_{C_i,C_l})$ based on collected data. To
estimate the probability density at a point $\mathbf{u} =
(d_{CA_i,C_l},d_{C_i,C_l}) \in \mathbb{R}^2$, we use the observed
$n$ pairs of data from the database $(\mathbf{x}_{1}, \cdots
\mathbf{x}_{n}) =((d_{CA_i,C_l,1},d_{C_i,C_l,1}), \cdots
(d_{CA_i,C_l,n},d_{C_i,C_l,n}))$ to derive the density function
$\pi(\mathbf{u})$, which takes the form of:
\begin{equation}
\left. \begin{aligned} \pi(\mathbf{u}) = \frac{1}{n}\sum
\limits_{i=1}^n
 |\mathbf{H}|^{-\frac{1}{2}} \mathbf{K}
[ \mathbf{H}^{-\frac{1}{2}} \cdot (\mathbf{u} - \mathbf{x}_{i})],\\
\end{aligned} \right.
\end{equation}
where $\mathbf{H}$ is the symmetric and positive definite bandwidth
$2\times2$ matrix, $\mathbf{K}$ is a bivariate gaussian kernel
function:
\begin{equation}
\left. \begin{aligned} \mathbf{K}(x) =
\frac{e^{(-\frac{1}{2}x^{T}x)}}{2\pi}.
\end{aligned} \right.
\end{equation}

To construct the bandwidth matrix $\mathbf{H}$, we calculate the
standard deviation $\sigma_{d_{CA_i,C_l}}$ of the $n$ pairs of
$(d_{CA_i,\, C_l}, d_{C_i,\, C_l})$. The corresponding entry
$h_{d_{CA_i,C_l}}$ in the bandwidth matrix $\mathbf{H}$ is set as
$h_{d_{CA_i,C_l}} =
\sigma_{d_{CA_i,C_l}}(\frac{1}{n})^{\frac{1}{6}}$. Similarly,
$h_{d_{C_i,C_l}}$ is set as $h_{d_{C_i,C_l}} =
\sigma_{d_{C_i,C_l}}(\frac{1}{n})^{\frac{1}{6}}$. The bandwidth
matrix $\mathbf{H}$ is then assembled as~\cite{bowman1997}:
\begin{equation}
\left. \begin{aligned} \mathbf{H} = \begin{pmatrix}
 h_{d_{CA_i,C_l}} & h_{d_{C_i,C_l}}\\
 h_{d_{C_i,C_l}} & h_{d_{CA_i,C_l}}\\
\end{pmatrix}.\\
\end{aligned} \right.
\end{equation}
We partition the domain of $(d_{CA_i,C_l}, \, d_{C_i,C_l})$ into a
grid with 32 grid points in each direction. $\pi(d_{CA_i,C_l},
d_{C_i,C_l})$ are estimated at the grid points, and interpolated by
a bilinear function elsewhere. Conditional distribution
$\pi(d_{C_i,C_l} | d_{CA_i,C_l})$ is constructed from the joint
distribution $\pi(d_{CA_i,C_l}, \, d_{C_i,C_l})$ when $d_{CA_i,C_l}$
is fixed. $d_{C_i,C_l}$ is sampled from $\pi(d_{C_i,C_l} |
d_{CA_i,C_l})$. We follow the same procedure to construct
$\pi(d_{N_{i+1},C_l} | d_{C_i, C_l})$, which is used to sample
$d_{N_{i+1},C_l}$.

\subsubsection*{Backbone dihedral angle distributions from the loop database}

Although the empirical conditional distributions can efficiently
guide chain growth to generate properly connected loop
conformations, the dihedral angles of the loops are often not
energetically favorable. As a result, conditional distributions
described above alone are not sufficient in generating near native
loop conformations.

The problem can be alleviated by an additional step of selecting a
subset of $n$ loops with low-energy dihedral angles from generated
samples. We use empirical distributions of the loop dihedral angles
obtained from the loop database. Specifically, for the $m$ sampled
positions of the current residue $i$ of type $\alpha_{i}$ with
dihedral angles $(\phi^{1}, \psi^{1}),..(\phi^{m},\psi^{m})$, we
select $n<m$ samples following an empirically derived backbone
dihedral angle distribution $\pi(\phi_i,\psi_i,\alpha_{i})$. Here
$\pi(\phi_i,\psi_i,\alpha_{i})$ is derived from the same protein
loop structure database for conditional distance distributions and
constructed by counting the frequencies of $(\phi, \, \psi)$ pairs
for each residue type.


\subsubsection*{Determining the number of trial states at each growth
step for backbone torsion angles}

It is important to determine the appropriate size of trial states
$m$ and $n$ for generating backbone conformations, as small $m$ and
$n$ values may lead to insufficient sampling, resulting in
inaccurate loop conformations. On the other hand, very large $m$ and
$n$ values will require significantly more computational time,
without significant gain in accuracy.

We use a data set, denoted as {\it tuning-set}\/ to determine the
optimal values of parameters $m$ and $n$ for sampling backbone
conformations. Part of this data set comes from that of Soto {\it et
al}~\cite{soto2008}. The rest are randomly selected from
pre-compiled CulledPDB (with $\le 20\%$ sequence identity, $\le 1.8$
\r{A}\ resolution, and $R\le 0.25$). It contains a total of $140$
loops, with $35$ loops of length 6, $35$ of length 8, $35$ of length
10, and $35$ of length 12.

The optimal values of $m$ and $n$ are determined as $(m = 160, n=
32)$ according to the test result on tuning-set
(Figure~\ref{fig:statechoose}).



\subsubsection*{Placement of backbone atoms}

From the $n$ sampled dihedral angle pairs $(\phi^{1},\psi^{1}),
\cdots, (\phi^{n},\psi^{n})$, we can calculate the coordinates of
atom $C_i$ and $N_{i+1}$ for all of the $n$ trials. $CA_{i+1}$ atoms
are sampled by generating random $\omega$ dihedral angles from a
normal distribution with mean $180^{\circ}$ and standard deviation
of $4^{\circ}$. Calculating the coordinates of backbone $O$ atoms
using standard bond length and angle values is straightforward.

The coordinates of backbone atoms of the $n$ samples at this
particular growth step can be denoted as $(\mathbf{x}^{1}_{C_i},
\mathbf{x}^{1}_{O_{i}}, \mathbf{x}^{1}_{N_{i+1}},
\mathbf{x}^{1}_{CA_{i+1}}, \cdots, \mathbf{x}^{k}_{C_i},
\mathbf{x}^{k}_{O_{i}}, \mathbf{x}^{k}_{N_{i+1}},
\mathbf{x}^{k}_{CA_{i+1}}, \cdots, \mathbf{x}^{n}_{C_i},
\mathbf{x}^{n}_{O_{i}}, \mathbf{x}^{n}_{N_{i+1}},
\mathbf{x}^{n}_{CA_{i+1}}, )$. For simplicity, we denote the
coordinates of the four atoms at residue $i$ as $S_i$ and the $k$-th
sample as $S^k_i$. We sample one of them using an energy criterion.
The probability for $S^k_i$ is defined by
$$
\pi_(S^k_i| S_t,S_{t+1},\cdots,S_{i-1}) \sim \exp({-E(S^k_i)}/T ),
$$ where $T=1$ is the effective temperature, and $E(S^k_i)$ is
the interaction energy of the four atoms defined by $S^k_i$ with the
remaining part of the protein, including those loop atoms sampled in
previous steps. The energy function $E$ is an atomic
distance-dependent empirical potential function constructed from the
loop database, which is effective in detecting steric clashes and
efficient to compute. Fragments with steric clashes are rarely drawn
because of their high energy values. In summary, the coordinates of
the four backbone atoms, $S_i = (C_i, O_i, N_{i+1}, CA_{i+1})$, is
drawn from the following joint distribution at this step:

\begin{equation}
    \left. \begin{aligned}
    S_i\sim \pi(d_{C_i,C_l} | d_{CA_i,C_l}) \cdot \pi(d_{N_{i+1},C_l} | d_{C_i, C_l}) \cdot \pi(\omega) \cdot \pi(\phi^{i},\psi^{i},\alpha_{i})\cdot  \pi(S_i| S_t,S_{t+1},\cdots,S_{i-1})
.
    \end{aligned} \right.
\end{equation}
Altogether, ($l-t)$ backbone dihedral angle combinations need to be
sampled. When the growing end is three residues away from the
$C$-terminal anchor atom of the loop, $C_l$, we apply the CSJD
analytical closure method to generate coordinates of the remaining
backbone atoms~\cite{coutsias2004}. Small fluctuations of bond
lengths, angles, and $\omega$ dihedral angles are introduced to the
analytical closure method to increase the success rate of loop
closure.

\subsection*{Improving computing efficiency}

To reduce computational cost of calculating atom-atom distances in
energy evaluation, we use a procedure, REsidue-residue Distance
Cutoff and ELLipsoid criterion ({\sc Redcell}) to reduce the amount
of computational time.


\subsubsection*{Residue-residue distance cutoff}

The residue-residue distance cutoff $d_{R}$ is used to exclude
residues far from the loop energy calculation. Instead of a
universal cutoff value, such as the $10$ \r{A}\ $C_{\beta} -
C_{\beta}$ distance used in reference~\cite{leaver2011}, we use a
residue-dependent distance cutoff value. The residue-residue
distance cutoff $d_R$ is assigned to be $r_{i} + r_{j} + c$, where
$r_i$ and $r_j$ are the effective radii of residue $i$ and $j$,
respectively. For one residue type, effective radii is the distance
between residue geometrical center and the heavy atom which is
farthest away from the residue geometrical center. $c$ is a constant
set to $8$ \r{A}. For a residue $i$ in the loop region and residue
$j$ in the non-loop region, we calculate the residue-residue
distance $d_{ij} = \|\mathbf{x}_i - \mathbf{x}_j\|$, where
$\mathbf{x}_i$ and $\mathbf{x}_j$ are the geometric centers of
residue $i$ and $j$, respectively. If $d_{ij} > d_R$, all of the
atoms in residue $j$ are excluded from energy calculation. This
residue-dependent cutoff is more accurate and ensures close residues
are included.

\subsubsection*{Ellipsoid criterion}

The basic idea of ellipsoid criterion is to construct a symmetric
ellipsoid such that all atoms that need to be considered for energy
calculation during loop sampling are enclosed in the ellipsoid.
Atoms that are outside of the ellipsoid can then be safely excluded.
The starting and ending residues of a loop naturally serve as the
two focal points of the ellipsoid. Intuitively, all backbone atoms
of a loop must be within an ellipsoid. Formally, we define a set of
points $\{\mathbf{x}\}$, the sum of whose distances to the two foci
is less than $L$, defined as the sum of the backbone bond lengths
$b_{C-C}$ of the loop of length $l$:
$$
\{\mathbf{x} = ({\it x_1}, {\it x_2}, {\it x_3}) \in \mathbb{R}^3|
\; \|\mathbf{x} - \mathbf{x}_1\| + \|\mathbf{x} - \mathbf{x}_2\| \le
L \},
$$
$$
L = 2a = \sum\limits^l b_{C-C},
$$
where $\mathbf{x}_1$ and $\mathbf{x}_2$  are the two focal points of
the ellipsoid. The symmetric ellipsoid ($b = c$) can be written as:
\begin{equation}
\label{eqn:ellipsoid}
    \left. \begin{aligned}
    \frac{{\it x_1}^2}{a^2} + \frac{{\it x_2}^2}{b^2} + \frac{{\it x_3}^2}{b^2} = 1,
    \end{aligned} \right.
\end{equation}
where $a = L/2$ and $b = [(L/2)^2 - (\frac{||\mathbf{x}_1 -
\mathbf{x}_2||}{2})^2]^{1/2}$ correspond to the semi-major axis and
semi-minor axis of the symmetric ellipsoid, respectively. To
incorporate the effects of side chain atoms, we enlarge the
ellipsoid by the amount of the maximum side-chain length $s$.
Furthermore, we assume that any atom can interact with a loop atom
if it is within a distance cut-off of $k$. As a result, the overall
enlargement of the ellipsoid is $(s + k)$. The final definition of
the enlarged ellipsoid for detecting possible atom-atom interactions
is given by Eqn~(\ref{eqn:ellipsoid}), with
\begin{equation}
a = (||\mathbf{x}_1 - \mathbf{x}_2||/2) \sec \alpha_2,
\label{eqn:ellipa}
\end{equation} and
\begin{equation}
b = (||\mathbf{x}_1 - \mathbf{x}_2||/2) \tan \alpha_1 + s + k,
\label{eqn:ellipb}
\end{equation}
where $\alpha_1$ is determined by the equation $\sec \alpha_1 =
\frac{L}{||\mathbf{x}_1 - \mathbf{x}_2||}$, and $\alpha_2$ by $\tan
\alpha_2 = \frac{(s+k) + (||\mathbf{x}_1 - \mathbf{x}_2||/2) \tan
  \alpha_1}{||\mathbf{x}_1 - \mathbf{x}_2||/2}$ (see Figure~\ref{fig:ellipsoid}B).

For any atom in the protein, if the sum of its distances to the two
foci points is greater than $2a$, this atom is permanently excluded
from energy calculations. The computational cost to enforce this
criterion depends only on the loop length and is independent of the
size the protein, once the rest of the residues have been examined
using the ellipsoid criterion. This improves our computing
efficiency significantly, especially for large proteins. This
criterion also helps to prune chain growth by terminating a growth
attempt if the placed atoms are outside the ellipsoid.


\subsection*{Side-chain modeling and steric clash removal}
\label{sidechain}

Side chains are built upon completion of backbone sampling of a
loop. For the $i$-th residue of type $a_i$, we denote the degrees of
freedom (DOFs) for its side chain as $s_{(a_i)}$. DOFs of side chain
residues depend on the residue types, {\it e.g.} Arg has four
dihedral angles ($\chi_{1}, \chi_{2}, \chi_{3}, \chi_{4}$), with
($s_{(ARG)} = 4$). Val only has one dihedral angle ($\chi_{1}$),
with ($s_{(VAL)} = 1$). Each DOFs is discretized into bins of
$4^{\circ}$, and only bins with non-zero entries for all loop
residues in the loop database are retained.

We sample $n_{sc}$ trial states of side chains from the empirical
distribution $\pi(\chi_{1} \cdots \chi_{s_{(a_i)}})$ obtained from
the loop database. One of $n_{sc}$ trials is then chosen according
to the probability calculated by the empirical potential. Denote the
side chain fragment for the $i$-th residue as $\mathbf{z}_i$, we
select $\mathbf{z}_{i}$ following the probability distribution:
$$
\pi_{i}(\mathbf{z}_{i}) \sim \exp({-E(\mathbf{z}_{i})}/T ),
$$ where $E (\mathbf{z}_{i}) $ is the interaction energy of the newly added
side chain fragment $\mathbf{z}_{i}$ with the remaining part of the
protein, and $T$ is the effective temperature.

When there are steric clashes between side chains, we rotate the
side-chain atoms along the $\mathrm{C_{\alpha}}-C_{\beta}$ axis for
all residue types except Pro. For Pro, we use the
$N-\mathrm{C_{\alpha}}$ axis for rotation. We consider two atoms to
be in steric clash if the ratio of their distance to the sum of
their van der Waals radii is less than $0.65$ ~\cite{jacobson2004}.


\subsection*{Potential function}

To evaluate the energy of loops, we developed a simple atom-based
distance-dependent empirical potential function, following
well-established
practices~\cite{sippl1990,miyazawa1996,lu2001,zhou2002,li2003,hu2004,zhang2005,shen2006,li2007}.
Empirical energy functions developed from databases have been shown
to be very effective in protein structure prediction, decoy
discrimination, and protein-ligand
interactions~\cite{samudrala1998,li2003,zhang2004b,zhang2005,zhang2004c,huang2006,huang2011,zimmermann2011}.
As our interest is modeling the loop regions, the atomic
distance-dependent empirical potential is built from loop structures
collected in the PDB~\cite{bernstein1977}.

Instead of using detailed $167$ atom types associated with the $20$
amino acids, we group all heavy atoms into $20$ groups, similar to
the approach used in Rosetta~\cite{sheffler2010}. The $16$
side-chain atom types comprise six carbon types, six nitrogen types,
three oxygen types, and one sulfur type. The $4$ backbone types are
N, $\mathrm{C_{\alpha}}$, C, and O. This simplified scheme helps to
alleviate the problem of sparsity of observed data for certain
parameter values. For an atom $i$ in the loop region of atom type
$a_{i}$ and an atom $j$ of atom type $a_{j}$, regardless whether $j$
is in the loop region, the distance-dependent interaction energy
$E_{(a_{i},a_{j};\:d_{ij})}$ is calculated as :
\begin{equation}
\left. \begin{aligned}
  E_{(a_{i},a_{j};\:d_{ij})} &=
         -\ln\frac{\pi(a_{i},a_{j};\:d_{ij})}{\pi^{\prime}(a_{i},a_{j};\:d_{ij})},\\
  \end{aligned} \right.
\end{equation}\\
where $E_(a_{i},a_{j};\:d_{ij})$ denotes the interaction energy
between a specific atom pair $(a_{i},a_{j})$ at distance $d_{ij}$,
$\pi(a_{i},a_{j};\:d_{ij})$ and $\pi^{\prime}(a_{i},a_{j};\:d_{ij})$
are the observed probability of this distance-dependent interaction
from the loop database and the expected probability from a random
model, respectively.

The observed probability $\pi(a_{i},a_{j};\:d_{ij})$ is calculated
as:
\begin{equation}
\left. \begin{aligned}
   \pi(a_{i},a_{j};\:d_{ij}) =
   \frac{n(a_{i},a_{j};\:d_{ij})}{n_{total}},
  \end{aligned} \right.
\end{equation}\\
where $n(a_{i},a_{j};\:d_{ij})$ is the observed count of $(a_i,
a_j)$ pairs found in the loop structures with the distance $d_{ij}$
falling in the predefined bins. We use a total of $60$ bins for
$d_{ij}$, ranging from $2$ \r{A}\ to $8$ \r{A}, with the bin width
set to $0.1$ \r{A}. $d_{ij}$ ranging from $0$ \r{A}\ to $2$ \r{A}\
is treated as one bin. Here $n(a_{i},a_{j};\:d_{ij}) =
\sum\limits_{k=1}^{N} n(a_{i},a_{j},\:d_{ij}(k))$, where $N$ is the
number of loops in our loop database, $n(a_{i},a_{j},\:d_{ij}(k))$
is the observed number of $(a_i, a_j)$ pairs at the distance of
$d_{ij}$ in the $k$-th loop. $n_{total}$ is the observed total
number of all atom pairs in the loop database regardless of the atom
types and distance, namely, $n_{total} = \sum\limits_{d_{ij}}
\sum\limits_{a_{j}} \sum\limits_{a_{i}} n(a_{i},a_{j};\:d_{ij})$.

The expected random distance-dependent probability of this pair
$\pi^{\prime}(a_{i},a_{j};\:d_{ij})$ is calculated based on sampled
loop conformations, called decoys. It is calculated as:
\begin{equation}
\left. \begin{aligned}
   \pi^{\prime}(a_{i},a_{j};\:d_{ij}) =
   \frac{n^{\prime}(a_{i},a_{j};\:d_{ij})}{n^{\prime}_{total}},
  \end{aligned} \right.
\end{equation}\\
where $n^{\prime}(a_{i},a_{j};\:d_{ij}) = \sum\limits_{k=1}^{N}
(\frac{\sum\limits_{x=1}^{M}
n^{\prime}(a_{i},a_{j},\:d_{ij}(x,k))}{M})$ is the expected number
of ($a_i, a_j;\:d_{ij})$ pairs averaged over all decoy loop
conformations of all target loops in the loop database. Here
$n^{\prime}(a_{i},a_{j},\:d_{ij}(x,k))$ is the number of $(a_i,
a_j)$ pairs at distance $d_{ij}$ in the $x$-th generated loop
conformations for the $k$-th loop. $M$ is the number of decoys
generated for a loop, which is set to $500$. $N$ is the number of
loops in our loop database. $n^{\prime}_{total}$ is the total number
of all atom pairs in the reference state, $n^{\prime}_{total} =
\sum\limits_{d_{ij}} \sum\limits_{a_{j}} \sum\limits_{a_{i}}
n^{\prime}(a_{i},a_{j};\:d_{ij})$.

\subsection*{Tool availability}
We have made the source code of {\sc DiSGro} available for download.
The URL is at: {\tt tanto.bioengr.uic.edu/DiSGro/}.

\section*{Acknowledgments}
We thank Drs.\ Youfang Cao, Joe Dundas, David Jimenez Morales,
Hammad Naveed, Hsiao-Mei Lu, and Gamze Gursoy, Meishan Lin, Yun Xu,
Jieling Zhao for helpful discussions.



\begin{thebibliography}{10}
\bibitem{bajorath1996}
Bajorath J, Sheriff S (1996) Comparison of an antibody model with an
x-ray
  structure: The variable fragment of {BR}96.
\newblock Proteins: Structure, Function, and Bioinformatics 24: 152--157.
\bibAnnoteFile{bajorath1996}

\bibitem{streaker1999}
Streaker E, Beckett D (1999) Ligand-linked structural changes in the
  escherichia coli biotin repressor: The significance of surface loops for
  binding and allostery.
\newblock Journal of molecular biology 292: 619--632.
\bibAnnoteFile{streaker1999}

\bibitem{myllykoski2012}
Myllykoski M, Raasakka A, Han H, Kursula P (2012) Myelin 2',
3'-cyclic
  nucleotide 3'-phosphodiesterase: active-site ligand binding and molecular
  conformation.
\newblock PloS one 7: e32336.
\bibAnnoteFile{myllykoski2012}

\bibitem{lotan2004}
Lotan I, Van Den~Bedem H, Deacon A, Latombe J (2004) Computing
protein
  structures from electron density maps: The missing loop problem.
\newblock In: Workshop on the Algorithmic Foundations of Robotics (WAFR). pp.
  153--68.
\bibAnnoteFile{lotan2004}

\bibitem{fiser2000}
Fiser A, Do R, {\v{S}}ali A (2000) Modeling of loops in protein
structures.
\newblock Protein science 9: 1753--1773.
\bibAnnoteFile{fiser2000}

\bibitem{sellers2008}
Sellers B, Zhu K, Zhao S, Friesner R, Jacobson M (2008) Toward
better
  refinement of comparative models: predicting loops in inexact environments.
\newblock Proteins: Structure, Function, and Bioinformatics 72: 959--971.
\bibAnnoteFile{sellers2008}

\bibitem{van1997}
van Vlijmen H, Karplus M (1997) {PDB}-based protein loop prediction:
parameters
  for selection and methods for optimization1.
\newblock Journal of molecular biology 267: 975--1001.
\bibAnnoteFile{van1997}

\bibitem{canutescu2003}
Canutescu A, Dunbrack~Jr R (2003) Cyclic coordinate descent: A
robotics
  algorithm for protein loop closure.
\newblock Protein Science 12: 963--972.
\bibAnnoteFile{canutescu2003}

\bibitem{de2003}
de~Bakker P, DePristo M, Burke D, Blundell T (2003) Ab initio
construction of
  polypeptide fragments: Accuracy of loop decoy discrimination by an all-atom
  statistical potential and the amber force field with the generalized born
  solvation model.
\newblock Proteins: Structure, Function, and Bioinformatics 51: 21--40.
\bibAnnoteFile{de2003}

\bibitem{depristo2003}
DePristo M, de~Bakker P, Lovell S, Blundell T (2003) Ab initio
construction of
  polypeptide fragments: efficient generation of accurate, representative
  ensembles.
\newblock Proteins: Structure, Function, and Bioinformatics 51: 41--55.
\bibAnnoteFile{depristo2003}

\bibitem{michalsky2003}
Michalsky E, Goede A, Preissner R (2003) Loops {I}n {P}roteins
({LIP})--a
  comprehensive loop database for homology modelling.
\newblock Protein engineering 16: 979--985.
\bibAnnoteFile{michalsky2003}

\bibitem{coutsias2004}
Coutsias E, Seok C, Jacobson M, Dill K (2004) A kinematic view of
loop closure.
\newblock Journal of computational chemistry 25: 510--528.
\bibAnnoteFile{coutsias2004}

\bibitem{jacobson2004}
Jacobson M, Pincus D, Rapp C, Day T, Honig B, et~al. (2004) A
hierarchical
  approach to all-atom protein loop prediction.
\newblock Proteins: Structure, Function, and Bioinformatics 55: 351--367.
\bibAnnoteFile{jacobson2004}

\bibitem{zhu2006}
Zhu K, Pincus D, Zhao S, Friesner R (2006) Long loop prediction
using the
  protein local optimization program.
\newblock Proteins: Structure, Function, and Bioinformatics 65: 438--452.
\bibAnnoteFile{zhu2006}

\bibitem{zhang2007a}
Zhang J, Kou S, Liu J (2007) Biopolymer structure simulation and
optimization
  via fragment regrowth monte carlo.
\newblock The Journal of chemical physics 126: 225101.
\bibAnnoteFile{zhang2007a}

\bibitem{cui2008}
Cui M, Mezei M, Osman R (2008) Prediction of protein loop structures
using a
  local move monte carlo approach and a grid-based force field.
\newblock Protein Engineering Design and Selection 21: 729--735.
\bibAnnoteFile{cui2008}

\bibitem{spassov2008}
Spassov V, Flook P, Yan L (2008) {LOOPER}: a molecular
mechanics-based
  algorithm for protein loop prediction.
\newblock Protein Engineering Design and Selection 21: 91--100.
\bibAnnoteFile{spassov2008}

\bibitem{liu2009}
Liu P, Zhu F, Rassokhin D, Agrafiotis D (2009) A self-organizing
algorithm for
  modeling protein loops.
\newblock PLoS computational biology 5: e1000478.
\bibAnnoteFile{liu2009}

\bibitem{hildebrand2009}
Hildebrand P, Goede A, Bauer R, Gruening B, Ismer J, et~al. (2009)
  Superlooper--a prediction server for the modeling of loops in globular and
  membrane proteins.
\newblock Nucleic acids research 37: W571--W574.
\bibAnnoteFile{hildebrand2009}

\bibitem{karmali2009}
Karmali A, Blundell T, Furnham N (2009) Model-building strategies
for
  low-resolution x-ray crystallographic data.
\newblock Acta Crystallographica Section D: Biological Crystallography 65:
  121--127.
\bibAnnoteFile{karmali2009}

\bibitem{mandell2009}
Mandell D, Coutsias E, Kortemme T (2009) Sub-angstrom accuracy in
protein loop
  reconstruction by robotics-inspired conformational sampling.
\newblock Nature methods 6: 551--552.
\bibAnnoteFile{mandell2009}

\bibitem{lee2010}
Lee J, Lee D, Park H, Coutsias E, Seok C (2010) Protein loop
modeling by using
  fragment assembly and analytical loop closure.
\newblock Proteins: Structure, Function, and Bioinformatics 78: 3428--3436.
\bibAnnoteFile{lee2010}

\bibitem{zhao2011}
Zhao S, Zhu K, Li J, Friesner R (2011) Progress in super long loop
prediction.
\newblock Proteins: Structure, Function, and Bioinformatics .
\bibAnnoteFile{zhao2011}

\bibitem{arnautova2011}
Arnautova Y, Abagyan R, Totrov M (2011) Development of a new
physics-based
  internal coordinate mechanics force field and its application to protein loop
  modeling.
\newblock Proteins: Structure, Function, and Bioinformatics 79: 477--498.
\bibAnnoteFile{arnautova2011}

\bibitem{goldfeld2011}
Goldfeld D, Zhu K, Beuming T, Friesner R (2011) Successful
prediction of the
  intra-and extracellular loops of four g-protein-coupled receptors.
\newblock Proceedings of the National Academy of Sciences 108: 8275--8280.
\bibAnnoteFile{goldfeld2011}

\bibitem{subramani2012}
Subramani A, Floudas C (2012) Structure prediction of loops with
fixed and
  flexible stems.
\newblock The Journal of Physical Chemistry B 116: 6670--6682.
\bibAnnoteFile{subramani2012}

\bibitem{fernandez2013}
Fernandez-Fuentes N, Fiser A (2013) A modular perspective of protein
  structures: application to fragment based loop modeling.
\newblock Methods in molecular biology (Clifton, NJ) 932: 141.
\bibAnnoteFile{fernandez2013}

\bibitem{bruccoleri1987}
Bruccoleri R, Karplus M (1987) Prediction of the folding of short
polypeptide
  segments by uniform conformational sampling.
\newblock Biopolymers 26: 137--168.
\bibAnnoteFile{bruccoleri1987}

\bibitem{zhang2006a}
Zhang J, Liu J (2006) On side-chain conformational entropy of
proteins.
\newblock PLoS computational biology 2: e168.
\bibAnnoteFile{zhang2006a}

\bibitem{zhang2007b}
Zhang J, Lin M, Chen R, Liang J, Liu J (2007) Monte carlo sampling
of
  near-native structures of proteins with applications.
\newblock PROTEINS: Structure, Function, and Bioinformatics 66: 61--68.
\bibAnnoteFile{zhang2007b}

\bibitem{rosenbluth1955}
Rosenbluth M, Rosenbluth A (1955) Monte carlo calculation of the
average
  extension of molecular chains.
\newblock The Journal of Chemical Physics 23: 356.
\bibAnnoteFile{rosenbluth1955}

\bibitem{grassberger1997}
Grassberger P (1997) Pruned-enriched rosenbluth method: Simulations
of $\theta$
  polymers of chain length up to 1 000 000.
\newblock Physical Review E 56: 3682.
\bibAnnoteFile{grassberger1997}

\bibitem{wong2013}
Wong SWK (2013) Statistical computation for problems in dynamic
systems and protein folding.
\newblock  PhD dissertation, Harvard University.
\bibAnnoteFile{wong2013}

\bibitem{liu1998}
Liu J, Chen R (1998) Sequential {M}onte {C}arlo methods for dynamic
systems.
\newblock Journal of the American statistical association : 1032--1044.
\bibAnnoteFile{liu1998}

\bibitem{liang2002}
Liang J, Zhang J, Chen R (2002) Statistical geometry of packing
defects of
  lattice chain polymer from enumeration and sequential monte carlo method.
\newblock The Journal of chemical physics 117: 3511.
\bibAnnoteFile{liang2002}

\bibitem{liu2008}
Liu J (2008) Monte Carlo strategies in scientific computing.
\newblock Springer Verlag.
\bibAnnoteFile{liu2008}

\bibitem{zhang2008}
Zhang J, Lin M, Chen R, Wang W, Liang J (2008) Discrete state model
and
  accurate estimation of loop entropy of {RNA} secondary structures.
\newblock The Journal of chemical physics 128: 125107.
\bibAnnoteFile{zhang2008}

\bibitem{zhang2004a}
Zhang J, Chen Y, Chen R, Liang J (2004) Importance of chirality and
reduced
  flexibility of protein side chains: A study with square and tetrahedral
  lattice models.
\newblock The Journal of chemical physics 121: 592.
\bibAnnoteFile{zhang2004a}

\bibitem{lin2008}
Lin M, Lu H, Chen R, Liang J (2008) Generating properly weighted
ensemble of
  conformations of proteins from sparse or indirect distance constraints.
\newblock The Journal of chemical physics 129: 094101.
\bibAnnoteFile{lin2008}

\bibitem{lin2011}
Lin M, Zhang J, Lu H, Chen R, Liang J (2011) Constrained proper
sampling of
  conformations of transition state ensemble of protein folding.
\newblock Journal of Chemical Physics 134: 75103.
\bibAnnoteFile{lin2011}

\bibitem{zhang2009}
Zhang J, Dundas J, Lin M, Chen R, Wang W, et~al. (2009) Prediction
of
  geometrically feasible three-dimensional structures of pseudoknotted {RNA}
  through free energy estimation.
\newblock RNA 15: 2248--2263.
\bibAnnoteFile{zhang2009}

\bibitem{soto2008}
Soto C, Fasnacht M, Zhu J, Forrest L, Honig B (2008) Loop modeling:
Sampling,
  filtering, and scoring.
\newblock Proteins: Structure, Function, and Bioinformatics 70: 834--843.
\bibAnnoteFile{soto2008}

\bibitem{cahill2003}
Cahill S, Cahill M, Cahill K (2003) On the kinematics of protein
folding.
\newblock Journal of computational chemistry 24: 1364--1370.
\bibAnnoteFile{cahill2003}

\bibitem{shenkin1987}
Shenkin P, Yarmush D, Fine R, Wang H, Levinthal C (1987) Predicting
antibody
  hypervariable loop conformation. i. ensembles of random conformations for
  ringlike structures.
\newblock Biopolymers 26: 2053--2085.
\bibAnnoteFile{shenkin1987}

\bibitem{xiang2002}
Xiang Z, Soto C, Honig B (2002) Evaluating conformational free
energies: the
  colony energy and its application to the problem of loop prediction.
\newblock Proceedings of the National Academy of Sciences 99: 7432--7437.
\bibAnnoteFile{xiang2002}

\bibitem{zhou2002}
Zhou H, Zhou Y (2002) Distance-scaled, finite ideal-gas reference
state
  improves structure-derived potentials of mean force for structure selection
  and stability prediction.
\newblock Protein Science 11: 2714--2726.
\bibAnnoteFile{zhou2002}

\bibitem{ko2011}
Ko J, Lee D, Park H, Coutsias E, Lee J, et~al. (2011) The
{FALC}-loop web
  server for protein loop modeling.
\newblock Nucleic acids research 39: W210--W214.
\bibAnnoteFile{ko2011}

\bibitem{simons1997}
Simons K, Kooperberg C, Huang E, Baker D, et~al. (1997) Assembly of
protein
  tertiary structures from fragments with similar local sequences using
  simulated annealing and bayesian scoring functions.
\newblock Journal of molecular biology 268: 209--225.
\bibAnnoteFile{simons1997}

\bibitem{rohl2004}
Rohl C, Strauss C, Misura K, Baker D, et~al. (2004) Protein
structure
  prediction using rosetta.
\newblock Methods in enzymology 383: 66.
\bibAnnoteFile{rohl2004}

\bibitem{sheffler2010}
Sheffler W, Baker D (2010) Rosettaholes2: A volumetric packing
measure for
  protein structure refinement and validation.
\newblock Protein Science 19: 1991--1995.
\bibAnnoteFile{sheffler2010}

\bibitem{leaver2011}
Leaver-Fay A, Tyka M, Lewis S, Lange O, Thompson J, et~al. (2011)
Rosetta3: an
  object-oriented software suite for the simulation and design of
  macromolecules.
\newblock Methods Enzymol 487: 545--574.
\bibAnnoteFile{leaver2011}

\bibitem{hu2004}
Hu C, Li X, Liang J (2004) Developing optimal non-linear scoring
function for
  protein design.
\newblock Bioinformatics 20: 3080--3098.
\bibAnnoteFile{hu2004}

\bibitem{thomas1996}
Thomas P, Dill K (1996) An iterative method for extracting
energy-like
  quantities from protein structures.
\newblock Proceedings of the National Academy of Sciences 93: 11628--11633.
\bibAnnoteFile{thomas1996}

\bibitem{huang2011}
Huang S, Zou X (2011) Statistical mechanics-based method to extract
atomic
  distance-dependent potentials from protein structures.
\newblock Proteins: Structure, Function, and Bioinformatics 79: 2648--2661.
\bibAnnoteFile{huang2011}

\bibitem{Engh1991}
Engh R, Huber R (1991) Accurate bond and angle parameters for x-ray
protein
  structure refinement.
\newblock Acta Crystallographica Section A: Foundations of Crystallography 47:
  392--400.
\bibAnnoteFile{Engh1991}

\bibitem{wang2003}
Wang G, Dunbrack R (2003) Pisces: a protein sequence culling server.
\newblock Bioinformatics 19: 1589--1591.
\bibAnnoteFile{wang2003}

\bibitem{kabsch1983}
Kabsch W, Sander C (1983) Dictionary of protein secondary structure:
pattern
  recognition of hydrogen-bonded and geometrical features.
\newblock Biopolymers 22: 2577--2637.
\bibAnnoteFile{kabsch1983}

\bibitem{lewis2008}
Lewis D (2008) Winsorisation for estimates of change.
\newblock SURVEY METHODOLOGY BULLETIN-OFFICE FOR NATIONAL STATISTICS- 62: 49.
\bibAnnoteFile{lewis2008}

\bibitem{bowman1997}
Bowman A, Azzalini A (1997) Applied smoothing techniques for data
analysis: the
  kernel approach with S-Plus illustrations, volume~18.
\newblock Oxford University Press, USA.
\bibAnnoteFile{bowman1997}

\bibitem{sippl1990}
Sippl M (1990) Calculation of conformational ensembles from
potentials of mena
  force.
\newblock Journal of molecular biology 213: 859--883.
\bibAnnoteFile{sippl1990}

\bibitem{miyazawa1996}
Miyazawa S, Jernigan R, et~al. (1996) Residue-residue potentials
with a
  favorable contact pair term and an unfavorable high packing density term, for
  simulation and threading.
\newblock Journal of molecular biology 256: 623--644.
\bibAnnoteFile{miyazawa1996}

\bibitem{lu2001}
Lu H, Skolnick J (2001) A distance-dependent atomic knowledge-based
potential
  for improved protein structure selection.
\newblock Proteins: Structure, Function, and Bioinformatics 44: 223--232.
\bibAnnoteFile{lu2001}

\bibitem{li2003}
Li X, Hu C, Liang J (2003) Simplicial edge representation of protein
structures
  and alpha contact potential with confidence measure.
\newblock Proteins: Structure, Function, and Bioinformatics 53: 792--805.
\bibAnnoteFile{li2003}

\bibitem{zhang2005}
Zhang J, Chen R, Liang J (2005) Empirical potential function for
simplified
  protein models: Combining contact and local sequence--structure descriptors.
\newblock Proteins: Structure, Function, and Bioinformatics 63: 949--960.
\bibAnnoteFile{zhang2005}

\bibitem{shen2006}
Shen M, Sali A (2006) Statistical potential for assessment and
prediction of
  protein structures.
\newblock Protein Science 15: 2507--2524.
\bibAnnoteFile{shen2006}

\bibitem{li2007}
Li X, Liang J (2007) Knowledge-based energy functions for
computational studies
  of proteins.
\newblock In: Computational methods for protein structure prediction and
  modeling, Springer. pp. 71--123.
\bibAnnoteFile{li2007}

\bibitem{samudrala1998}
Samudrala R, Moult J (1998) An all-atom distance-dependent
conditional
  probability discriminatory function for protein structure prediction.
\newblock Journal of molecular biology 275: 895--916.
\bibAnnoteFile{samudrala1998}

\bibitem{zhang2004b}
Zhang J, Chen R, Liang J (2004) Potential function of simplified
protein models
  for discriminating native proteins from decoys: Combining contact interaction
  and local sequence-dependent geometry.
\newblock In: Engineering in Medicine and Biology Society, 2004. IEMBS'04. 26th
  Annual International Conference of the IEEE. IEEE, volume~2, pp. 2976--2979.
\bibAnnoteFile{zhang2004b}

\bibitem{zhang2004c}
Zhang C, Liu S, Zhou Y (2004) Accurate and efficient loop selections
by the
  {DFIRE}-based all-atom statistical potential.
\newblock Protein science 13: 391--399.
\bibAnnoteFile{zhang2004c}

\bibitem{huang2006}
Huang S, Zou X (2006) An iterative knowledge-based scoring function
to predict
  protein--ligand interactions: I. derivation of interaction potentials.
\newblock Journal of computational chemistry 27: 1866--1875.
\bibAnnoteFile{huang2006}

\bibitem{zimmermann2011}
Zimmermann M, Leelananda S, Gniewek P, Feng Y, Jernigan R, et~al.
(2011) Free
  energies for coarse-grained proteins by integrating multibody statistical
  contact potentials with entropies from elastic network models.
\newblock Journal of structural and functional genomics 12: 137--147.
\bibAnnoteFile{zimmermann2011}

\bibitem{bernstein1977}
Bernstein F, Koetzle T, Williams G, Meyer~Jr E, Brice M, et~al.
(1977) The
  protein data bank: a computer-based archival file for macromolecular
  structures.
\newblock Journal of molecular biology 112: 535--542.
\bibAnnoteFile{bernstein1977}

\end{thebibliography}



\newpage
\section*{Figure Legends}








\section*{Supporting Information Legends}


\newpage
\section*{Tables}












\end{document}
